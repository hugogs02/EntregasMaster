\documentclass[a4paper,onecolumn]{article}
\usepackage[page,toc,titletoc,title]{appendix}
\usepackage{url}
\usepackage{subfigure}
\usepackage[sc]{mathpazo} % Use the Palatino font
\usepackage[T1]{fontenc} % Use 8-bit encoding that has 256 glyphs
\usepackage[utf8]{inputenc} % Use utf-8 as encoding
\linespread{1.05} % Line spacing - Palatino needs more space between lines
\usepackage{microtype} % Slightly tweak font spacing for aesthetics
\usepackage[spanish, activeacute]{babel}
 \decimalpoint
% \usepackage[hmarginratio=1:1,top=32mm,columnsep=20pt]{geometry} % Document marginshttps://www.overleaf.com/project/60211b96f72a79d4c7515e93
% \usepackage[hang, small,labelfont=bf,up,textfont=it,up]{caption} % Custom captions under/above floats in tables or figures
\usepackage{verbatim} % comentarios
\usepackage{listings}
\usepackage{xcolor}
\lstset{
    inputencoding=utf8,
    language=SQL,
    frame=single,
    basicstyle=\ttfamily\small,
    keywordstyle=\color{blue}\bfseries,
    identifierstyle=\color{black},
    commentstyle=\color{gray}\itshape,
    stringstyle=\color{red},
    numbers=left,
    numberstyle=\tiny\color{gray},
    stepnumber=1,
    numbersep=10pt,
    showspaces=false,
    showstringspaces=false,
    breaklines=true,
    breakindent=0pt,
    breakatwhitespace=false,
    tabsize=2,
    captionpos=b
}
\setlength{\parskip}{0.8em}
\usepackage{natbib}
\usepackage{enumitem}
% \setlist[itemize]{noitemsep} % Make itemize lists more compact
% \usepackage{abstract} % Allows abstract customization
% \renewcommand{\abstractnamefont}{\normalfont\bfseries} % Set the "Abstract" text to bold
% \renewcommand{\abstracttextfont}{\normalfont\small\itshape} % Set the abstract itself to small italic text
\usepackage{titlesec}

\usepackage{fancyhdr} % Headers and footers
\pagestyle{fancy} % All pages have headers and footers
\fancyhead{}
\lhead{Hugo Gómez Sabucedo}
\rhead{Minería de datos y modelización predictiva}

\renewcommand{\footrulewidth}{0.2pt}
\usepackage{titling} % Customizing the title section
\usepackage[breaklinks=true]{hyperref} % For hyperlinks in the PDF
%\usepackage{array}
%\newcolumntype{C}[1]{>{\centering\let\newline\\\arraybackslash\hspace{0pt}}m{#1}}
\usepackage{graphicx}
%\usepackage{lipsum} % NO NECESARIO LUEGO
%\usepackage{amsmath}
%\usepackage{wrapfig}
%\usepackage{multicol}
%\usepackage{bm}


\let\stdsection\section
\renewcommand\section{\newpage\stdsection}

%-------------------------------------------------------------------------------
%	TITLE SECTION
%-------------------------------------------------------------------------------

\setlength{\droptitle}{-4\baselineskip} % Move the title up



\title{\begin{center} \Huge Minería de datos y modelización predictiva \end{center}} % Article title
\author{
    \textsc{\Huge Hugo Gómez Sabucedo} \\ % Your name
    \large \href{mailto:hugogomezsabucedo@gmail.com}{hugogomezsabucedo@gmail.com} \\ [2ex] % Your email address
    \Large \textbf{Máster Big Data, Data Science \& Inteligencia Artificial} \\
    \normalsize Curso 2024-2025 \\
    \large Universidad Complutense de Madrid
}
\date{} % Leave empty to omit a date

\begin{document}
% Print the title
\maketitle
\tableofcontents
\begin{sloppypar}

%-------------------------------------------------------------------------
%	DOCUMENT
%-------------------------------------------------------------------------

\section{Introducción} \label{enunciado}
En esta práctica se nos pide, a partir de un archivo con datos sobre diferentes resultados electorales, seleccionar unas variables, con el objetivo
de construir tanto un modelo de regresión lineal, a partir de una variable objetivo continua; como un modelo de regresión logística, a partir de una
variable binaria. Para ello, se deberán asignar correctamente los tipos de los datos y realizar un análisis de los mismos, con el objetivo inicial 
de depurarlos, para asegurarnos que no tenemos variables incoherentes o que no se ajusten al modelo. A continuación, se corregirán los errores que se 
hayan detectado, así como los valores atípicos y perdidos. Una vez con los datos limpios, podremos analizar las relaciones entre las distintas variables 
input y las variables objetivo, para poder proceder con la creación de los dos modelos solicitados, para cada una de las variables.

Los 

Todas las variables correcatmente asignadas.
Frecuencia en categoricas. algunos ayuntamieentos repetidos, tiene sentido (mieres en cataluña y asturias).
CCAA:hay mas valores en unas que otras,pero tiene sentido. actividadppal agrupamos construccion e industria en otros. vemos valores perdidos en densidad (?)
abstencion alta, izquierda y derecha dos valores: normal. hay valores perdidos en todos los casos, pero puede deberse a perdidos o a valores repetidos

variables numericas. todas tienen suficientes valores distintos.
poblacion:aplicamos transformacion logaritmica, ya que tienen un rango muy alto debido a municipios que tienen mucha poblacion. lo mismo con censo, inmueble y pob2010.
ageover65pt, foreignerspt tienen valores minimos negativos. deben eliminarse, no es posible. age1965pt, samecomautpt tienen>100. imposible.
explotaciones tiene max 99999. fuera de rango? tratar
tenemos varios nan: totalempresas, industria, construccion, comercttehost, servicios, inmuebles, pob2010, pobchange_pct (estos dos 7 los dos), personasinmueble. tratar

definimos como identificadora NAme y codprovincia

relacion entre valores perdidos: inmuebles y personasinmueble, pob2010 y pobchange, totalcensus y personasinmueble
casi todas las variables tienen valores perdidos. los analizamos. ninguna tiene mas de un 50% (max 9%)
la observacion que mas variables perdidas tiene es un 34por ciento, pero no se elimina (el minimo es 50). pero transformamos la variable a categoricas

datos faltantes en totalEmpresas, industria, construccion, comercttehosteleria, servicios, inmuebles, pob2010, superficie, pobchangeÑ_pct, personasinmueble


\appendix
%\section{Anexo: Script de SQL}\label{anexo1}
%\lstinputlisting{HugoGomezSabucedo.sql}

\end{sloppypar}
\end{document}
