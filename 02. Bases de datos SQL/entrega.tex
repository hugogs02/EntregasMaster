\documentclass[a4paper,onecolumn]{article}
\usepackage[toc,page]{appendix}
\usepackage{url}
\usepackage{subfigure}
\usepackage[sc]{mathpazo} % Use the Palatino font
\usepackage[T1]{fontenc} % Use 8-bit encoding that has 256 glyphs
\usepackage[utf8]{inputenc} % Use utf-8 as encoding
\linespread{1.05} % Line spacing - Palatino needs more space between lines
\usepackage{microtype} % Slightly tweak font spacing for aesthetics
\usepackage[spanish, activeacute]{babel}
\decimalpoint
\usepackage{amsmath}
\usepackage{amssymb}
\usepackage{amsfonts}
\usepackage{appendix}
\usepackage[hmarginratio=1:1,top=32mm,columnsep=20pt]{geometry} % Document marginshttps://www.overleaf.com/project/60211b96f72a79d4c7515e93
\usepackage[hang, small,labelfont=bf,up,textfont=it,up]{caption} % Custom captions under/above floats in tables or figures
\usepackage{booktabs} % Horizontal rules in tables
\usepackage{verbatim} % comentarios
\usepackage{listings}
\lstset{
    frame=single,
    breaklines=true,
    numbers=left,
    keywordstyle=\color{blue},
    numbersep=5pt,
    numberstyle=,
    basicstyle=\linespread{1.5}\selectfont\ttfamily,
    commentstyle=\color{gray},
    stringstyle=\color{orange},
    identifierstyle=\color{green!40!black},
}
\lstdefinestyle{console}
{
    numbers=left,
    backgroundcolor=\color{violet},
    %%belowcaptionskip=1\baselineskip,
    breaklines=true,
    %%xleftmargin=\parindent,
    %%showstringspaces=false,
    basicstyle=\footnotesize\ttfamily,
    %%keywordstyle=\bfseries\color{green!40!black},
    %%commentstyle=\itshape\color{green},
    %%identifierstyle=\color{blue},
    %%stringstyle=\color{orange},
    basicstyle=\scriptsize\color{white}\ttfamily,
}
\setlength{\parskip}{0.8em}
\usepackage{natbib}
\usepackage{enumitem}
\setlist[itemize]{noitemsep} % Make itemize lists more compact
\usepackage{abstract} % Allows abstract customization
\renewcommand{\abstractnamefont}{\normalfont\bfseries} % Set the "Abstract" text to bold
\renewcommand{\abstracttextfont}{\normalfont\small\itshape} % Set the abstract itself to small italic text
\usepackage{titlesec}

\usepackage{fancyhdr} % Headers and footers
\pagestyle{fancy} % All pages have headers and footers
\fancyhead{}
\lhead{Hugo Gomez Sabucedo}
\rhead{Bases de datos SQL}

\renewcommand{\footrulewidth}{0.2pt}
\usepackage{titling} % Customizing the title section
\usepackage[breaklinks=true]{hyperref} % For hyperlinks in the PDF
\usepackage{array}
\newcolumntype{C}[1]{>{\centering\let\newline\\\arraybackslash\hspace{0pt}}m{#1}}
\usepackage{graphicx}
\usepackage{lipsum} % NO NECESARIO LUEGO
\usepackage{xcolor} % NO NECESARIO LUEGO
\usepackage{amsmath}
\usepackage{wrapfig}
\usepackage{multicol}
\usepackage{bm}

\usepackage[page,toc,titletoc,title]{appendix}

\let\stdsection\section
\renewcommand\section{\newpage\stdsection}
%-------------------------------------------------------------------------------
%	TITLE SECTION
%-------------------------------------------------------------------------------

\setlength{\droptitle}{-4\baselineskip} % Move the title up

\pretitle{\huge\bfseries} % Article title formatting
\posttitle{} % Article title closing formatting




\title{\begin{center} \Huge Bases de datos SQL \end{center}} % Article title
\author{
	\textsc{\LARGE Hugo Gómez Sabucedo} \\ % Your name
	\large \href{mailto:hugogomezsabucedo@gmail.com}{hugogomezsabucedo@gmail.com} \\ [2ex] % Your email address
	\large \textbf{Máster Big Data, Data Science & Inteligencia Artificial} \\
        \normalsize Curso 2024-2025 \\
        \small Universidad Complutense de Madrid
}
\date{} % Leave empty to omit a date

\begin{document}
% Print the title
\maketitle
\tableofcontents
%\begin{sloppypar}

%-------------------------------------------------------------------------
%	DOCUMENT
%-------------------------------------------------------------------------

\section{Formulación do problema} \label{formulacion}
O problema que se nos presenta é o seguinte:

Un grupo de ladróns están planeando o roubo dun banco e a súa posterior fuxida en coche polas rúas da cidade. Para isto, desexan chegaar o máis rápido posible dende o banco ó garaxe no que planean esconderse co botín. Dado o seguinte rueiro, no que en cada rúa figuran os minutos estimados en percorrela, que ruta deberían escoller para minimizar dita distancia? Canto duraría a fuxida en dito caso?

\begin{figure}[h]
%\includegraphics[width=\textwidth]{PFCM.png}
\centering
\end{figure}

\section{Resolución mediante o algoritmo de Dijkstra}\label{algoritmo}
Para a resolución do problema, faremos uso do algoritmo de Dijkstra, o cal se atopa en \cite[páx.142]{Libro}.

\subsection{Paso 1}\label{paso1}
O primeiro paso do algoritmo é definir o conxunto $X$ (que representa os nodos visitados) que será $X=\emptyset$ e o conxunto $\overline{X}$ (que representa os nodos que quedan por visitar), que será $\overline{X}=N\backslash X=N$. A continuación, definiremos dous vectores, do mesmo tamaño que o número de nodos do noso problema (é dicir, 15): o vector $\pi$ (que representa a lonxitude do camiño máis corto entre o nodo $1$ e o nodo $i$), cuxo valor inicial será $\infty$ para todo $i \in N$; e o vector $e$ (que representa o nodo predecesor ó nodo $i$), con valor $-1$. Así, temos que:
\begin{itemize}
    \item $X=\emptyset$
    \item $\overline{X}=\{1, 2, 3, 4, 5, 6, 7, 8, 9, 10, 11, 12, 13, 14, 15\}$
    \item $\pi = (\infty, \infty, \infty, \infty, \infty, \infty, \infty, \infty, \infty, \infty, \infty, \infty, \infty, \infty, \infty )$
    \item $e = (-1, -1, -1, -1, -1, -1, -1, -1, -1, -1, -1, -1, -1, -1, -1 )$
\end{itemize}

\subsection{Paso 2}\label{paso2}
O segundo paso do algoritmo consiste en engadir o nodo orixe, 1, ó conxunto $X$, e actualizar o seu $\pi_1$ e $e_1$ a 0. Ademais, para cada $i\in\overline{X}$, actualizamos $\pi_i=c_{1i}$ e $e_i=1$. No noso caso, dende o nodo 1 só podemos ir ós nodos 2 e 3, ambos con custo 8; polo que $\pi_2=\pi_3=8$ e $e_1=e_2=1$. Desta forma, os conxuntos quedan como segue:
\begin{itemize}
    \item $X=\{1\}$
    \item $\overline{X}=\{2, 3, 4, 5, 6, 7, 8, 9, 10, 11, 12, 13, 14, 15\}$
    \item $\pi = (0, 8, 8, \infty, \infty, \infty, \infty, \infty, \infty, \infty, \infty, \infty, \infty, \infty, \infty )$
    \item $e = (0, 1, 1, -1, -1, -1, -1, -1, -1, -1, -1, -1, -1, -1, -1 )$
\end{itemize}

\subsection{Paso 3}\label{paso3}
O paso 3 consiste en 3 subpasos, que se repetirán ata chegar á solución óptima. 
\begin{enumerate}[label=\roman*)]
    \item Encontrar o $i \in \overline{X}$ que cumpra que o seu $\pi_i$ sexa o mínimo entre o de tódolos $i \in \overline{X}$.
    \item Engadir o nodo $i$ a $X$, actualizando $X=X\cup\{i\}$ e $\overline{X}=N\backslash X$.
    \item Actualízase o vector $\pi$. Para tódolos $j\in\overline{X}$, calculamos se $\pi_j=\pi_i+c_{ij}$ (onde $c_{ij}$ é o custo de ir de $i$ a $j$) é menor que o $\pi_j$ que xa tiñamos e, se se cumpre, actualizamos o $pi_j$ con dito valor e $e_j=i$.
    \item Se $X=N$, o algoritmo rematou, tendo como solución óptima o $\pi_n$, e coa ruta que indican os predecesores; en caso contrario, repetimos o paso 3.
\end{enumerate}

Unha vez aplicado isto, procedemos coa primeira iteración.

\newpage
\begin{appendices}
\section{Anexo I: Script de SQL}\label{apendice}

\lstset{language=R}
%\lstinputlisting[frame=single]{Tarefa4_PFCM_GomezSabucedoHugo.r}

\end{appendices}

\newpage
\begin{thebibliography}{X}
    \bibitem{Libro} 
       \textsc{GONZÁLEZ DÍAZ, Julio}. \textit{Modelos y Ténicas de Optimización} [en liña]. 2023 [consultado o 28 de abril de 2023]. Dispoñible en: \url{https://cv.usc.es/pluginfile.php/1790406/mod_resource/content/2/Ap_MTO_ETSE.pdf}
\end{thebibliography} 

\end{sloppypar}
\end{document}