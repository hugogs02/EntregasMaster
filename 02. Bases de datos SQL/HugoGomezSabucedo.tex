\documentclass[a4paper,onecolumn]{article}
\usepackage[page,toc,titletoc,title]{appendix}
\usepackage{url}
\usepackage{subfigure}
\usepackage[sc]{mathpazo} % Use the Palatino font
\usepackage[T1]{fontenc} % Use 8-bit encoding that has 256 glyphs
\usepackage[utf8]{inputenc} % Use utf-8 as encoding
\linespread{1.05} % Line spacing - Palatino needs more space between lines
\usepackage{microtype} % Slightly tweak font spacing for aesthetics
\usepackage[spanish, activeacute]{babel}
 \decimalpoint
% \usepackage[hmarginratio=1:1,top=32mm,columnsep=20pt]{geometry} % Document marginshttps://www.overleaf.com/project/60211b96f72a79d4c7515e93
% \usepackage[hang, small,labelfont=bf,up,textfont=it,up]{caption} % Custom captions under/above floats in tables or figures
\usepackage{verbatim} % comentarios
\usepackage{listings}
\usepackage{xcolor}
\lstset{
    inputencoding=utf8,
    language=SQL,
    frame=single,
    basicstyle=\ttfamily\small,
    keywordstyle=\color{blue}\bfseries,
    identifierstyle=\color{black},
    commentstyle=\color{gray}\itshape,
    stringstyle=\color{red},
    numbers=left,
    numberstyle=\tiny\color{gray},
    stepnumber=1,
    numbersep=10pt,
    showspaces=false,
    showstringspaces=false,
    breaklines=true,
    breakindent=0pt,
    breakatwhitespace=false,
    tabsize=2,
    captionpos=b
}
\setlength{\parskip}{0.8em}
\usepackage{natbib}
\usepackage{enumitem}
% \setlist[itemize]{noitemsep} % Make itemize lists more compact
% \usepackage{abstract} % Allows abstract customization
% \renewcommand{\abstractnamefont}{\normalfont\bfseries} % Set the "Abstract" text to bold
% \renewcommand{\abstracttextfont}{\normalfont\small\itshape} % Set the abstract itself to small italic text
\usepackage{titlesec}

\usepackage{fancyhdr} % Headers and footers
\pagestyle{fancy} % All pages have headers and footers
\fancyhead{}
\lhead{Hugo Gómez Sabucedo}
\rhead{Bases de datos SQL}

\renewcommand{\footrulewidth}{0.2pt}
\usepackage{titling} % Customizing the title section
\usepackage[breaklinks=true]{hyperref} % For hyperlinks in the PDF
%\usepackage{array}
%\newcolumntype{C}[1]{>{\centering\let\newline\\\arraybackslash\hspace{0pt}}m{#1}}
\usepackage{graphicx}
%\usepackage{lipsum} % NO NECESARIO LUEGO
%\usepackage{amsmath}
%\usepackage{wrapfig}
%\usepackage{multicol}
%\usepackage{bm}


\let\stdsection\section
\renewcommand\section{\newpage\stdsection}

%-------------------------------------------------------------------------------
%	TITLE SECTION
%-------------------------------------------------------------------------------

\setlength{\droptitle}{-4\baselineskip} % Move the title up



\title{\begin{center} \Huge Bases de datos SQL \end{center}} % Article title
\author{
    \textsc{\Huge Hugo Gómez Sabucedo} \\ % Your name
    \large \href{mailto:hugogomezsabucedo@gmail.com}{hugogomezsabucedo@gmail.com} \\ [2ex] % Your email address
    \Large \textbf{Máster Big Data, Data Science \& Inteligencia Artificial} \\
    \normalsize Curso 2024-2025 \\
    \large Universidad Complutense de Madrid
}
\date{} % Leave empty to omit a date

\begin{document}
% Print the title
\maketitle
%tableofcontents
\begin{sloppypar}

%-------------------------------------------------------------------------
%	DOCUMENT
%-------------------------------------------------------------------------

\section{Enunciado del problema} \label{enunciado}
El enunciado del problema es el siguiente:

Tenemos una empresa dedicaba a la organización de eventos culturales únicos “ArteVida Cultural”. Organizamos desde deslumbrantes
conciertos de música clásica hasta exposiciones de arte vanguardista, pasando por apasionantes obras de teatro y cautivadoras
conferencias, llevamos la cultura a todos los rincones de la comunidad.

Necesitamos gestionar la gran variedad de eventos y detalles, así como las ganancias que obtenemos. Para ello, es
necesario llevar un registro adecuado de cada evento, de los artistas que los protagonizan, las ubicaciones donde
tienen lugar, la venta de entradas y, por supuesto, el entusiasmo de los visitantes que asisten.

Hemos decidido diseñar e implementar una base de datos relacional que no solo simplifique la organización de
eventos, sino que también permita analizar datos valiosos para tomar decisiones informadas.

En nuestra empresa ofrecemos una serie de actividades que tienen un nombre, un tipo: concierto de distintos tipos
de música (clásica, pop, blues, soul, rock and roll, jazz, reggaeton, góspel, country, …), exposiciones, obras de teatro
y conferencias, aunque en un futuro estamos dispuestos a organizar otras actividades. Además, en cada actividad
participa uno o varios artistas y un coste (suma del caché de los artistas).

El artista tiene un nombre, un caché que depende de la actividad en la que participe y una breve biografía.

La ubicación tendrá un nombre (Teatro Maria Guerrero, Estadio Santiago Bernabeu, …), dirección, ciudad o pueblo,
aforo, precio del alquiler y características.

De cada evento tenemos que saber el nombre del evento (p.e. “VI festival de música clásica de Alcobendas”), la
actividad, la ubicación, el precio de la entrada, la fecha y la hora, así como una breve descripción del mismo. En un
evento sólo se realiza una actividad.

También tendremos en cuenta los asistentes a los eventos, de los que sabemos su nombre completo, sus teléfonos
de contacto y su email. Una persona puede asistir a más de un evento y a un evento pueden asistir varias personas.

Nos interesará realizar consultas por el tipo de actividad, en que fecha se han realizado más eventos, en qué ciudad
realizamos más eventos, …

\section{Diseño conceptual} \label{mer}
El diagrama entidad-relación que se propone es el siguiente:
\begin{center}
    \begin{figure}[h!]
        \includegraphics[width=\textwidth]{MER.png}
    \end{figure}
\end{center}

A continuación se explica más en profundidad dicho diagrama y las diferentes entidades y relaciones que lo componen.

En resumen, se han definido seis entidades (actividad, tipoActividad, artista, evento, localización y persona), cada una de las 
cuales se corresponde con los principales conceptos que se mencionan en el enunciado del ejercicio. Además, se han definido
cinco relaciones (sucedeEn, tiene, actúaEn, acudeA y esDeTipo) entre las entidades anteriormente mencionadas.

En primer lugar tenemos la entidad \textbf{Evento}, que representa cada uno de los eventos que realiza la empresa. Esta entidad tendrá como atributos
el nombre, la fecha y hora (que en este punto he considerado como dos atributos diferenciados, aunque a posteriori en el paso a tablas pueda
definirse como un único atributo, por ejemplo, de tipo \textit{timestamp}), el precio de la entrada y la descripción del mismo. Además, se ha
añadido un atributo adicional, \underline{id\_evento}, el cual se usará como clave primaria de la entidad. Aunque incialmente había pensado en 
utilizar como clave del evento su nombre y la fecha y hora, no me parece la mejor decisión, también en terminos de rendimiento.

Un evento tiene lugar en una ubicación en concreto. Es por eso que se define la entidad \textbf{Localización}, que representa las diferentes
salas o lugares donde puede tener lugar un evento. Como se indica, sus atributos son su nombre, la dirección, la ciudad, el aforo,
el precio del alquiler y las características del mismo. A mayores, se ha definido un atributo adicional \underline{id\_localizacion},
el cual, al igual que en el caso anterior, usaremos como clave primaria de la entidad.

Entre estas dos entidades surge la relación \textbf{sucedeEn}, que es de 1 a muchos entre Evento y Localización, ya que un evento sólamente puede
tener lugar en una localización, pero una misma localización puede albergar varios eventos a lo largo del tiempo.

Por una parte, se define la entidad \textbf{Persona}, que representa los diferentes asistentes a los eventos. De éstas, únicamente se nos indica 
que nos interesan su nombre completo, su email y su teléfono. El nombre completo lo hemos definido como un atributo compuesto, el cual dividimos
a su vez en nombre, apellido1 y apellido2. Por su parte, como clave primaria se ha considerado el \underline{email} y el \underline{teléfono} de la 
persona. En el enunciado del problema no se especifica que se vayan a recoger identificadores únicos de las personas, como podría ser el DNI, por lo que
este no se puede añadir como atributo. Además, emplear un identificador autogenerado y autoincremental me parecía redundante respecto al resto de 
entidades, y poco eficiente. Por eso he considerado el email y teléfono como la clave primaria, ya que me parece que es una combinación que, en cualquier
caso, debería ser única (no tendría mucho sentido que una persona se registrase varias veces pero con diferentes combinaciones de email y teléfono).

Esta entidad tiene únicamente una relación, \textbf{acudeA}, que la relaciona con el evento. Esta relación es de muchos a muchos, ya que se indica que una persona
puede asistir a varios eventos, y que a un evento asisten varias personas. Además, en el enunciado se comenta que para los eventos se desea conocer
''el entusiasmo de los visitantes que asisten''. Es por ello que se ha definido un atributo de esta relación, valoración, que nos permitirá registrar
las opiniones de los asistentes a los diferentes eventos.

Por otra parte, se define la entidad \textbf{Actividad}, que se corresponde con cada una de las diferentes actividades que tienen lugar en el evento.
Como se indica, de una actividad nos interesa su nombre, su tipo y su coste. Adicionalmente, se define el atributo \underline{id\_actividad}, para 
establecerlo como clave primaria de la entidad. Respecto al coste de la actividad, se indica que el coste es la suma del caché de los artistas. 
Este coste, en el modelo entidad relación, se ha definido como un atributo diferenciado, pero es importante destacar que depende de la relación entre
actividad y artista (un artista, por ejemplo, puede tener un caché diferente para diferentes actividades). En el paso a tablas se 
explicará como se abordará esta cuestión en nuestra base de datos. La actividad se relaciona con evento mediante la relación \textbf{tiene}, de 1 a 
muchos entre evento y actividad. Es decir, un evento sólo tiene una actividad, pero una actividad puede estar en varios eventos.

Respecto a la actividad, se indica que puede ser de diferentes tipos. A priori, podría parecer lógico definir el tipo de la actividad como un atributo 
de la propia entidad. Sin embargo, parece más adecuado definir el tipo de actividad como una entidad diferenciada, ya que los tipos de actividad 
serán siempre los mismo (dentro de un conjunto), y esto nos permite además definir atributos adicionales como por ejemplo una breve descripción. Es 
por esto que nace la entidad \textbf{tipoActividad}, cuyos atributos son un \underline{id\_tipo}, así como el nombre o tipo y la descripción. Esta 
entidad se relaciona con Actividad mediante la relación \textbf{esDeTipo}, que es de cardinalidad varios a uno. Es decir, una actividad sólamente es 
de un tipo en concreto, pero un tipo puede tener varias actividades asociadas.

Por último, se define la entidad \textbf{Artista}, que únicamente tiene como atributos el nombre del mismo y su biografía, así como un atributo 
adicional \underline{id\_artista} que se usará como clave primaria. El Artista se relaciona con la Actividad (ya que se indica en el enunciado que 
un artista participa en una actividad, y no en un evento), mediante la relación \textbf{actúaEn}. Esta relación tiene un atributo, caché, que 
representa el caché del artista. Como se comentó anteriormente, el caché de un artista no es una cantidad fija, sino que dependiendo de la actividad, 
un artista puede tener un caché diferente. Por eso, no puede considerarse un atribto de la entidad artista, ya que no es algo intrínseco a él, 
sino que se establece al crearse la actividad en cuestión. Es por eso que se ha decidido incluir como un atributo de la relación actúaEn. Además, esta 
relación es de muchos a muchos: se indica que en cada actividad participan uno o varios artistas y, naturalmente, un artista puede participar
en múltiples actividades.


\section{Diseño lógico} \label{mr}
Una vez que tenemos definidas todas las entidades, podemos realizar el diseño lógico de la base de datos, mediante el paso a tablas del modelo 
entidad relación para obtener el modelo relacional.

En primer lugar, deberemos crear una tabla por cada una de las entidades que tenemos en el MER. Esto nos dará lugar a seis tablas, con sus 
correspondientes atributos: evento, localización, persona, actividad, tipoActividad y artista.

Por otra parte, aquellas relaciones que son de muchos a muchos generarán una tabla. De esta forma, se añaden dos nuevas tablas: acudeA y actúaEn. 
En cada una de estas tablas, guardaremos las claves primarias de cada una de las entidades participantes, así como los atributos, si los hubiese,
de dicha relación. Para las que son de 1 a muchos, se pasará la clave primaria de la entidad con participación 1 a la entidad con participación n.

Como detalle, destacar que los atributos \textit{fecha} y \textit{hora} de la entidad Evento se han fusionado en uno solo, fecha, ya que a nivel 
de implementación en SQL se puede definir como un tipo \textit{timestamp}, facilitando el trabajo con los datos.

De esta forma, el modelo relacional es el siguiente:

\noindent
\textbf{LOCALIZACION}(\underline{id\_localizacion}, nombre, direccion, ciudad, aforo, alquiler, caracteristicas)\\ \\
\textbf{EVENTO}(\underline{id\_evento}, \textit{id\_localizacion}, fecha, precio\_entrada, descripción)\\ \\
\textbf{ACTIVIDAD}(\underline{id\_actividad}, \textit{id\_evento}, \textit{id\_tipo}, nombre, coste)\\ \\
\textbf{TIPOACTIVIDAD}(\underline{id\_tipo}, tipo, descripcion)\\ \\
\textbf{ARTISTA}(\underline{id\_artista}, nombre, biografia)\\ \\
\textbf{PERSONA}(\underline{email}, \underline{telefono}, nombre, apellido1, apellido2)\\ \\
\textbf{ACTUAEN}(\textit{id\_artista}, \textit{id\_actividad}, cache)\\ \\
\textbf{ACUDEA}(\textit{email}, \textit{telefono}, \textit{id\_evento}, valoracion)\\ \\


\section{Implementación y consultas} \label{sql}
El script completo de la implementación de la base de datos se encuenta en el Anexo \ref{anexo1}. Sin embargo, aquí se 
explicará la elección de algunos tipos de los datos, así como un breve detalle de los triggers y vistas definidos y de las consultas propuestas, 
aclarando brevemente su objetivo y como se llega al mismo.

Lo primero que debemos hacer es comprobar que la base de datos cumple las tres formas normales:
\begin{itemize}
\item Respecto a la Primera Forma Normal (\textbf{FN1}), esta establece que no debe haber grupos repetidos de columnas ni una columna con múltiples
valores. Esto se comprueba viendo que todas las tablas tienen atributos atómicos y una clave primaria que asegura la unicidad de los registros.
\item La Segunda Forma Normal (\textbf{FN2}) dice que, además de cumplirse la FN1, debe haber dependencia funcional: los atributos que no forman
parte de ninguna clave deben depender completamente de la clave primaria. Esto se cumple viendo que todas las tablas con claves compuestas
(\texttt{ActuaEn} y \texttt{AcudeA}) sus atributos dependen de toda la clave (en caso de ActuaEn, cache depende del id\_artista e id\_actividad;
y en el caso de AcudeA, valoracion depende de email, telefono e id\_evento); y para el resto, las claves primarias son simples
\item Por último, la Tercera Forma Normal (\textbf{FN3}) establece que, además de cumplirse la FN2, no debe haber ninguna dependencia transitiva; 
es decir, cada columna debe estar relacionada directamente con las columnas de la clave primaria. Esto se comprueba directamente, por ejemplo, en 
la tabla Persona, donde vemos que nombre, apellido1 y apellido2 dependen directamente de la clave compuesta (email, telefono).
\end{itemize}

Una vez comprobamos las formas normales, pasamos a explicar brevemente algunos detalles sobre la implementación en SQL de la base de datos. En 
general, se han elegido los tipos de datos en función del atributo y su naturaleza, siendo por ejemplo la mayor parte de las claves números enteros,
puesto que las definimos como un id correlativo. Valores como el alquiler de una localización, el precio de una entrada de un evento o el coste de una 
actividad se han definido como \texttt{DECIMAL}, con un total de 10 dígitos de los cuales 2 son decimales. Para los valores de tipo \texttt{varchar}, 
sus longitudes se han determinado en función del posible valor que almacenarán; por su parte, valores como la biografía del artista o las caracaterísticas 
de una localización se han definido como tipo \texttt{TEXT} para que se pueda almacenar una mayor cantidad de información en los mismos.

Respecto a las tablas, para casi todas aquellas claves foráneas se ha definido una política de borrado y actualización en cascada, para que al eliminar 
algún registro de una tabla, se elimine también de la tabla dependiente. Únicamente en el caso de la tabla Actividad se ha definido que, en caso de que 
se elimine un tipo de actividad, ese campo se ponga a null en la tabla de Actividad, ya que creo que tiene más sentido en este caso guardar un valor nulo.

En lo que respecta a los triggers, se han creado dos, uno de los cuales está duplicado para que se ejecute tanto tras un update como tras un insert: 
el primero de ellos nos servirá para actualizar el coste de una actividad. Este valor, por defecto, lo hemos establecido a 0 (se supone que, cuando 
definamos inicialmente una actividad, quizás no sabremos aún los artistas). Pero, como indicamos, el coste depende del caché de los artistas, y este 
puede o bien variar (ya que un artista puede aumentar o disminuir su caché puntualmente) o simplemente aumentar debido a que se sumen más artistas. Por tanto,
se han creado dos triggers idénticos (\texttt{actualiza\_coste\_actividad\_insert} y \texttt{actualiza\_coste\_actividad\_update}), que se ejecutarán tras cada 
insert o update en la tabla ActuaEn, que actualizarán el coste de la actividad correspondiente, estableciéndolo como la suma de los cachés de las actividades 
que coincidan, empleando un coalesce(caché, 0) para evitar posibles errores. 

Por otra parte, tenemos el trigger \texttt{valida\_aforo}, el cual nos sirve para comprobar que no se exceda el aforo de un recinto. En este caso, se 
ejecuta tras cada insert en AcudeA, y lo que hace es, por una parte, obtener el aforo máximo de la localización donde tiene lugar el evento al que acude la 
persona en cuestión; y por otra, obtener el aforo actual o número de personas que tiene ese evento. Si el aforo actual fuese mayor o igual que el máximo, 
no nos dejaría hacer el insert, y mostraría un mensaje por pantalla. Se muestra un ejemplo de funcionamiento de este caso, aunque comentado, para permitir 
que el script se ejecute correctamente por completo.

En lo que respecta a las vistas, se ha creado una sencilla, \texttt{ValoracionesEventos}, que nos permite ver, para cada evento, la valoración que le han 
dado los asistentes, así como el total de personas que han asistido a ese evento. Además, se ha creado la vista \texttt{GananciasEvento}, que nos permite 
obtener las ganancias de cada evento, teniendo en cuenta que las ganancias son la suma de todas las entradas vendidas en ese evento, restando el coste 
de todas las actividades y el alquiler del recinto.

Las consultas que se han propuesto son las siguientes:
\begin{enumerate}
    \item Total de eventos por ciudad: obtener el número total de eventos en cada ciudad.
    \item Sabiendo el coste promedio de las actividades, seleccionar aquellas que superen dicho coste promedio.
    \item Obtener las ganancias de cada uno de los artistas en las actividades que han participado con nuestra empresa, ordenado de forma descendente.
    \item Obtener los dos eventos menos populares, es decir, aquellos con menor asistencia.
    \item Obtener el promedio de valoración que han otorgado las diferentes personas a los eventos que han asistido, con dos dígitos decimales.
    \item Obtener el top 3 de eventos que más actividades han tenido.
    \item Conocer cuáles son los eventos más costosos.
    \item Averiguar el porcentaje de ocupación de cada evento.
\end{enumerate}

\newpage

\appendix
\section{Anexo: Script de SQL}\label{anexo1}
\lstinputlisting{HugoGomezSabucedo.sql}

\end{sloppypar}
\end{document}
