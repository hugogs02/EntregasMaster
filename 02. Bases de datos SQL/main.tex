\documentclass[a4paper,onecolumn]{article}
\usepackage[page,toc,titletoc,title]{appendix}
\usepackage{url}
\usepackage{subfigure}
\usepackage[sc]{mathpazo} % Use the Palatino font
\usepackage[T1]{fontenc} % Use 8-bit encoding that has 256 glyphs
\usepackage[utf8]{inputenc} % Use utf-8 as encoding
\linespread{1.05} % Line spacing - Palatino needs more space between lines
\usepackage{microtype} % Slightly tweak font spacing for aesthetics
\usepackage[spanish, activeacute]{babel}
 \decimalpoint
% \usepackage[hmarginratio=1:1,top=32mm,columnsep=20pt]{geometry} % Document marginshttps://www.overleaf.com/project/60211b96f72a79d4c7515e93
% \usepackage[hang, small,labelfont=bf,up,textfont=it,up]{caption} % Custom captions under/above floats in tables or figures
\usepackage{verbatim} % comentarios
\usepackage{listings}
\lstset{
frame=single,
breaklines=true,
numbers=left,
keywordstyle=\color{blue},
numbersep=5pt,
numberstyle=,
basicstyle=\linespread{1.5}\selectfont\ttfamily,
commentstyle=\color{gray},
stringstyle=\color{orange},
identifierstyle=\color{green!40!black},
}
\lstdefinestyle{console}
{
numbers=left,
backgroundcolor=\color{violet},
%belowcaptionskip=1\baselineskip,
breaklines=true,
%xleftmargin=\parindent,
%showstringspaces=false,
basicstyle=\footnotesize\ttfamily,
%keywordstyle=\bfseries\color{green!40!black},
%commentstyle=\itshape\color{green},
%identifierstyle=\color{blue},
%stringstyle=\color{orange},
basicstyle=\scriptsize\color{white}\ttfamily,
}
\setlength{\parskip}{0.8em}
\usepackage{natbib}
\usepackage{enumitem}
% \setlist[itemize]{noitemsep} % Make itemize lists more compact
% \usepackage{abstract} % Allows abstract customization
% \renewcommand{\abstractnamefont}{\normalfont\bfseries} % Set the "Abstract" text to bold
% \renewcommand{\abstracttextfont}{\normalfont\small\itshape} % Set the abstract itself to small italic text
\usepackage{titlesec}

\usepackage{fancyhdr} % Headers and footers
\pagestyle{fancy} % All pages have headers and footers
\fancyhead{}
\lhead{Hugo Gómez Sabucedo}
\rhead{Bases de datos SQL}

\renewcommand{\footrulewidth}{0.2pt}
\usepackage{titling} % Customizing the title section
\usepackage[breaklinks=true]{hyperref} % For hyperlinks in the PDF
% \usepackage{array}
% \newcolumntype{C}[1]{>{\centering\let\newline\\\arraybackslash\hspace{0pt}}m{#1}}
% \usepackage{graphicx}
% \usepackage{lipsum} % NO NECESARIO LUEGO
% \usepackage{xcolor} % NO NECESARIO LUEGO
% \usepackage{amsmath}
% \usepackage{wrapfig}
% \usepackage{multicol}
% \usepackage{bm}


\let\stdsection\section
\renewcommand\section{\newpage\stdsection}

%-------------------------------------------------------------------------------
%	TITLE SECTION
%-------------------------------------------------------------------------------

\setlength{\droptitle}{-4\baselineskip} % Move the title up



\title{\begin{center} \Huge Bases de datos SQL \end{center}} % Article title
\author{
    \textsc{\Huge Hugo Gómez Sabucedo} \\ % Your name
    \large \href{mailto:hugogomezsabucedo@gmail.com}{hugogomezsabucedo@gmail.com} \\ [2ex] % Your email address
    \Large \textbf{Máster Big Data, Data Science \& Inteligencia Artificial} \\
    \normalsize Curso 2024-2025 \\
    \large Universidad Complutense de Madrid
}
\date{} % Leave empty to omit a date

\begin{document}
% Print the title
\maketitle
\tableofcontents
\begin{sloppypar}

%-------------------------------------------------------------------------
%	DOCUMENT
%-------------------------------------------------------------------------

\section{Enunciado del problema} \label{enunciado}
El enunciado del problema es el siguiente:

Tenemos una empresa dedicaba a la organización de eventos culturales únicos “ArteVida Cultural”. Organizamos desde deslumbrantes \
conciertos de música clásica hasta exposiciones de arte vanguardista, pasando por apasionantes obras de teatro y cautivadoras \
conferencias, llevamos la cultura a todos los rincones de la comunidad.

Necesitamos gestionar la gran variedad de eventos y detalles, así como las ganancias que obtenemos. Para ello, es\
necesario llevar un registro adecuado de cada evento, de los artistas que los protagonizan, las ubicaciones donde\
tienen lugar, la venta de entradas y, por supuesto, el entusiasmo de los visitantes que asisten.

Hemos decidido diseñar e implementar una base de datos relacional que no solo simplifique la organización de\
eventos, sino que también permita analizar datos valiosos para tomar decisiones informadas.

En nuestra empresa ofrecemos una serie de actividades que tienen un nombre, un tipo: concierto de distintos tipos\
de música (clásica, pop, blues, soul, rock and roll, jazz, reggaeton, góspel, country, …), exposiciones, obras de teatro\
y conferencias, aunque en un futuro estamos dispuestos a organizar otras actividades. Además, en cada actividad\
participa uno o varios artistas y un coste (suma del caché de los artistas).

El artista tiene un nombre, un caché que depende de la actividad en la que participe y una breve biografía.

La ubicación tendrá un nombre (Teatro Maria Guerrero, Estadio Santiago Bernabeu, …), dirección, ciudad o pueblo,\
aforo, precio del alquiler y características.

De cada evento tenemos que saber el nombre del evento (p.e. “VI festival de música clásica de Alcobendas”), la\
actividad, la ubicación, el precio de la entrada, la fecha y la hora, así como una breve descripción del mismo. En un\
evento sólo se realiza una actividad.

También tendremos en cuenta los asistentes a los eventos, de los que sabemos su nombre completo, sus teléfonos\
de contacto y su email. Una persona puede asistir a más de un evento y a un evento pueden asistir varias personas.

Nos interesará realizar consultas por el tipo de actividad, en que fecha se han realizado más eventos, en qué ciudad\
realizamos más eventos, …

\section{Diseño conceptual} \label{mer}
O problema que se nos presenta é o seguinte:


\section{Diseño lógico} \label{mr}

\section{Implementación y consultas} \label{sql}


\end{sloppypar}
\end{document}

\newpage
\begin{appendices}
\section{Anexo I: Script de SQL}\label{apendice}

\lstset{language=SQL}
%\lstinputlisting[frame=single]{HugoGomezSabucedo.sql}

\end{appendices}
